%%%%%%%%%%%%%%%%%%%%%%%%%%%%%%%%%%%%%%%%%%%%%%%%%%%%%%%%%%%%%%%%%%%%%%%%%%%%%%%%%%%%
% Template for STAT 548 Qualifying Paper Report
% Author: Ben Bloem-Reddy <benbr@stat.ubc.ca>
% Revised: Daniel J. McDonald <daniel@stat.ubc.ca>
% Date: 26 August 2024
%%%%%%%%%%%%%%%%%%%%%%%%%%%%%%%%%%%%%%%%%%%%%%%%%%%%%%%%%%%%%%%%%%%%%%%%%%%%%%%%%%%%

% Note: You will get an empty bibliography warning when compiling until you include a citation.

\documentclass[12pt]{article}
\usepackage[commenters={OA,DJM}]{shortex} % replace OA with your initials
\usepackage[round]{natbib}
\usepackage[margin=2.5cm]{geometry}
\newcommand{\email}[1]{\href{mailto:#1}{#1}}

% minor adjustments to ShorTeX
\let\argmin\relax\DeclareMathOperator*{\argmin}{argmin}
\let\argmax\relax\DeclareMathOperator*{\argmax}{argmax}
\DeclareMathOperator*{\minimize}{minimize}
\DeclareMathOperator*{\maximize}{maximize}
\DeclareMathOperator{\subjto}{\ \text{subject to}\ }
\renewcommand{\top}{\mathsf{T}}
\renewcommand{\d}{\mathsf{d}}

\graphicspath{{fig/}}



%%%%%%%%%%%%%%%%%%%%%%%%%%%%%%%%%%%%%%%%%%%%%%%%%%%%%%%%%%%%%%%%%%%%%%%%%%%%%%%%%

% your title/author/date information go here
\title{\LaTeX\ Template for STAT 548 Qualifying Paper Report} % replace with your title, a meaningful title
\author{Daniel J. McDonald} % replace with your name
\date{\today} % replace with your submission date


% start of document
\begin{document}
\maketitle

You can remove everything from here to the end of the document when you start
writing your report.

But do read through it first to get an idea of how to use the style file and 
some best practices for writing the report.

\section{Requirements}\label{sec:req}

As described on my website
\url{https://dajmcdon.github.io/teaching/stat548.html}, the report should
contain three main sections:

\bdesc
\item[1. Summary ($\sim$3 pages)] The first section of the report should provide a
summary of the paper and the problem(s) it addresses, including its relationship
to any previous work, its major contributions (e.g., novel techniques,
algorithmic developments, problem formulations, theoretical contributions), and
any limitations or shortcomings (e.g., restrictive assumptions, computational
constraints, flawed methodology). The aim of this section is for you to
synthesize the findings of a body of work and clearly present the important
points.

\item[2. Mini-proposal(s) for research projects] Each proposal should describe a
research project that applies, extends, generalizes, adapts, or addresses
shortcomings of the QP. Seemingly unrelated ideas inspired by the original QP
are also fine. You may write more than one proposal, but you must write at least
one. A proposal should concisely describe: the primary problem to be addressed;
an approach (or multiple approaches) for addressing the problem; any technical
or conceptual sub-problems; the potential impact of the project. You are not
expected to pursue any of these projects (though we can talk more if you would
like to). The aim of this section is to get you thinking creatively about
research, and to begin developing the skills necessary for writing research
proposals. Each proposal should be no more than \textbf{2 pages max}.

\item[3. QP specific project results] Each potential QP listed below has a brief
description of a related project. We will discuss the project in detail in our
initial meeting, and we can meet again (as many times as necessary) before the
report due date. Your grade will not be affected by how good the results look,
whether your approach improves on past work, or whether you achieve the initial
goal of the project. I will use this project to evaluate your research 
potential, which includes (among other aspects):
  \bitem
  \item clearly formulating a research question;
  \item setting up a useful mathematical framework for the problem;
  \item thinking creatively and independently to develop a solution;
  \item relating the problem to existing work, in other fields if necessary;
  \item being resourceful and asking questions when necessary;
  \item learning from and moving past the inevitable setbacks;
  \item reformulating the research problem when necessary;
  \item implementing new methods in code (when applicable);
  \item choosing appropriate experiments and metrics;
  \item communicating and reflecting on progress, setbacks, and results;
  \item thinking of future research directions.
  \eitem
\edesc



The report should be submitted as a GitHub repository based on the template
\href{https://github.com/dajmcdon/qp-template}{here}. The template includes a LaTeX
style file that should be used for the report. (Detailed instructions for usage
can be found in the repository's README file.) Any experimental/numerical
results should be reproducible. All code should be reusable, clearly
commented/documented, and exist in the \texttt{src/} folder of the same GitHub
repository to which you give me access as a collaborator. Code can be in any
language you wish, though my strong preference is for \texttt{R} or
\texttt{python}.


The remainder of this document demonstrates the style file and provides some
best practices for writing the report. 

\section{Introduction}\label{sec:intro}

We write some math for fun:

\[
\label{eq:1}
\int_{-\infty}^{\infty} \frac{1}{\sqrt{2\pi}|\Sigma|^{n/2}}
\exp\cbra*{-\frac{1}{2}(y - \mu)\T \Sigma^{-1} (y-\mu)} \d y = 1.
\]

We encourage the use of the various macros defined in
\href{https://github.com/trevorcampbell/shortex/}{ShorTeX},
so do your best. It makes things easier to read, but also provides lots of
necessary mathematics definitions that render nicely. 

Be careful with things like KL divergence and conditional probability
statements. I find that 
\[\KL(q\ \Vert\ p)\]
looks much better than 
\[\KL(q||p),\] 
and I similarly prefer 
\[
Y\ |\ X \sim \Norm(X,\ \sigma^2) 
\quad\textrm{to}\quad Y | X \sim \Norm(X,\sigma^2).
\] 
Note that the reals are $\reals^p$. There is also
\[
\shbeta = \argmin_{\beta \in \reals^p} \frac{1}{2n}\norm{\sky - \skX\beta}_2^2 + 
\lambda \norm{\beta}_1,
\]
and (with automatic sizing of the norm),
\[
\shbeta = \argmin_{\beta \in \reals^p} \frac{1}{2n}\norm*{\frac{y}{1} - X\beta}_2^2 + 
\lambda \norm*{\beta}_1.
\]

Note that the indicator function looks like $\1\{\cdot\}$, but I sometimes
prefer $\mathbf{1}\{\cdot\}$.




\subsection{Cleveref}\label{sec:clever}

We prefer to use cleveref to get nice references to things. For example, you can
say that \cref{eq:1} was printed in \cref{sec:intro}. No need to write out
things like ``Section''.

\section{Some best practices}

Some of these are taken from
\href{http://faculty.marshall.usc.edu/Jacob-Bien/papers/manuscript-checklist.pdf}{Jacob
Bien}. Note the use of the ShorTeX itemize environment style below.

\bitem   
\item Section titles should be all title case or all sentence case. Don't mix and match.
\item I prefer data set to dataset.
\item I prefer data to be singular. There remains debate on this point. When you
use the word ``datum'' in a sentence, then we can argue. Data is a mass noun,
like ``information''. We don't say ``How many data are enough?'', we say ``How much
data is enough.'' Enough said.  
\item Terminology is lower case, unless it's a person's name: \Nystrom extension
and lasso.
\item Equations are parts of the sentence. Displayed equations almost always
have a comma or period after. Very rarely is there a colon or comma \emph{before}
a displayed equation.
\item Don't start sentences with math (``$\Sigma$ is the covariance of
$\skX$.'') or the name of a software package that's lowercase, \eg
``\texttt{glmnet} is my favourite software''.
\item Don't use contractions.
\item No need to put dollar signs around numbers: 12 versus $12$.
\item DO put dollar signs around math: $p$ not p.
\item Use $x\gg y$ not $x >>y$.
\item Careful with parentheticals and references. Wrong: (see, \eg
\citet{Akaike1973}). Right: (see, \eg \citealp{Akaike1973}).
\item Never use \texttt{eqnarray}, always use \texttt{align}. Note that ShorTeX
makes \verb+\[   \]+ into an align environment, so you can just use that always.
\item For editing purposes, it is much better if the text is hard-wrapped rather
than soft wrapped.
\item asdfasdfsadf $\alpha \beta \hat{\gamma}$
\eitem


\subsection{Tables}

\Cref{tab:1} is a nice looking table. Strive for these.

\btab
\bcent
\btabr{@{}lr@{}}
\toprule
Ingredient & Quantity\\
\midrule
Fusili & 100 g\\
Eggs & 2\\
Salt & 1 tsp\\
Guanciale & 50 g\\
Pepper & \nicefrac{1}{2} tsp\\
Grated parmesan & \nicefrac{1}{4} c\\
\bottomrule
\etabr
\caption{This is a nice looking table. It might make carbonara.}
\label{tab:1}
\ecent
\etab


\section{Discussion} 

We made amazing contributions to the world of musical fractal pasta 
\citep{McDonald2017,Tibshirani2013}. We use Natbib, so be sure to use
\citep{Stein1981} for parenthetical references. Or you can say, according to
\citet{HastieTibshirani2009}, we should strive to balance truth and lies.


\bibliographystyle{rss}
\bibliography{qp.bib}

\end{document}
